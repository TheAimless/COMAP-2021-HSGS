\date{\today}
\currentvolume{1}
\currentissue{1}

\SetPaperName{Solar Times:}

\SetHeaderName{Solar Times}

\SetPaperLocation{Florida}
\SetPaperSlogan{``Powered by solar power.''}
\SetPaperPrice{Zero Dollars}
\maketitle
\thispagestyle{fancy}

\begin{multicols}{3}

\headline{Solar batteries: Factors to consider when installing}

As solar power becomes increasingly popular as an alternative energy source, more people are having trouble installing a solar system, especially those who go off-grid. There are many factors to consider; however, one of the most important parts of your home's electrical system is the energy storage system: batteries. These devices provide a way to store your hard-earned solar energy, in addition to powering all the necessary electric appliances in your house. Therefore, it is crucial to understand all the details involved when installing a battery storage system.

First, there are the energy demands within your household. Every piece of electrical equipment requires electricity in one way or another, so if you plan to go solar, you need to calculate your total energy usage per day in kilowatt-hours (kWh). By calculating the power consumed and the number of hours used by each electric appliance in your house, you can measure the total energy required to power your home.

After doing the calculations, you need to choose a battery type for your array. The two most popular choices for general home usage are lead-acid batteries and lithium-ion batteries. While lead-acid batteries have existed for a long time and are reliable, lithium-ion batteries not only offer you more storage per dollar you spend but also utilise solar energy with higher efficiency and depth of discharge (the percentage of power you consume from the battery) without quick degradation.

Next, you can start estimating the number of batteries required for your house. Either by using a solar battery calculator or calculating them by hand, you can adjust the number of batteries to purchase. You can also do this with other battery products to find the minimum cost of installation, which can save extra money.

In the end, after picking your desirable battery type, you can either hire an electrician to install your home circuit or rely on Internet guides to do that yourself. 
\closearticle

\headline{Cement batteries: A new future?}

Researchers from Gothenburg, Sweden have developed a battery model that incorporates the most common building material: cement. This allows previously unused wall space to be utilised as batteries, which may potentially revolutionise battery technology. By assessing the combinations of many common metals, the team has developed multiple batteries with metals embedded inside a concrete slab. After several testings, the researchers discovered that cement battery with nickel anode and iron cathode gives the best result: an average energy density of 7 Wh/m$^{2}$, or 0.8 Wh/L. Despite having a lower energy density than traditional solar batteries, researchers Emma Zhang and Luping Tang are optimistic about the possibility of "building rechargeable cement-based batteries on a large scale, with regards to the huge volume of a building." 

\closearticle

\end{multicols}
\restoregeometry