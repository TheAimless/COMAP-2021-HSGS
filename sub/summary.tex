\afterpage{
\newgeometry{left=0.5in,right=0.5in,top=0.2in,bottom=0.5in}
\begin{centering}
    Team Control Number

    {\Huge 11682}

    Problem Chosen

    {\Huge \textbf{A}}

    {\Large \textbf{2021}}\\
    {\textbf{HiMCM/MidMCM}}\\
    {\textbf{Summary Sheet}}\\~\\

\end{centering}
\makebox[\textwidth]{\rule{\textwidth}{0.6pt}}
\begin{centering}
    (Your team's summary should be included as the first page of your electronic submission.)\\
    Type a summary of your results on this page. Do not include the name of your school, advisor, or team members on this page.

\end{centering}
Recently, solar power has become a popular alternative energy source. As more people begin the transition to renewable sources of energy, the need for more accessible information steadily rises. The typical solar energy systems includes four crucial parts : the panels, inverters, racking and solar battery storage units. However, installing battery banks for use when there is no sunlight remains a big problem for homeowners. Choosing the wrong type of battery leads to either insufficient wattage output, which makes generators or an external power grid necessary, or too much money spent on solar batteries. As such, those intending not to rely on the local electrical grid need to determine the number and type of battery they should install depending on their needs and budgets. Our mathematical model aims to address the following problem, establish the most effective way to set up battery arrays, in addition to analysing cement batteries. 

To start with, we identified a list of electrical appliances, along with their power ratings and the average number of hours used for each appliance. Each power rating is then multiplied by the number of hours used, before summing up to get the total energy used per day. 

Then, to find the "best" battery to use in a battery storage system, our team divided the model into two parts. The first part contains information about multiple batteries, with each battery having different electrical specifications. This section aims to establish aspects of the battery that needs to be taken into consideration. The second part finds the "best" battery to be used in a battery array through a rating system: multiple factors are evaluated, with each factor having different weights. These factors are then combined into a general rating, which allows the assessment of various batteries. 

As different types of batteries possess distinctive specifications, batteries having different chemical compositions and specifications can be compared using our method. Therefore, this model can be extended to any type of battery, hence it is easily adjustable to personal needs. For instance, different lengths of autonomy, energy consumption and state of health return different results when charting each battery’s rating on a bar chart. However, the model should only be consulted as reference material, and not a definitive tool to purchase a particular type of battery to install in a photovoltaic system. 

Finally, a recently developed technological advancement known as cement battery is addressed and evaluated after our model. We explored multiple advantages and disadvantages of the battery; in addition, we also designed a way to incorporate the battery in the house and other factors to consider before we can apply our mathematical model to the battery. By assessing the cement battery thoroughly, our team can evaluate its potential if it is used for solar energy storage, as cement batteries can be installed in walls, saving up space normally used by currently available batteries.
\clearpage
\restoregeometry
}