After analysing requirement (1.a) thoroughly, we have constructed our answer to the energy demands in the 1600-square-foot home. There are many factors to consider; hence, before making a list of electrical appliances, we need to answer several questions in regards to the electrical needs of the family members residing in the house:
\begin{enumerate}
    \item The number of family members living in the house
\end{enumerate}
As the average size of an American household is four people, we have decided that the family will consist of two parents, a son and a daughter. This will most likely be the household that consumes the most electricity, as each person will require different energy demands in the house.
\begin{enumerate}[resume]
    \item The charge/discharge cycle of the batteries
\end{enumerate}
Our team assumed that the batteries start charging when there is no one in the house; only appliances that constantly operate which are fridges and security cameras are powered. The batteries will stop charging when everyone is at home using electrical appliances. In addition, we assume that everyone will either start going to school or commuting to work at 8:30 a.m., and everyone is presented at home at 5:30 p.m. when the sun begins to set during the winter months. Therefore, the average number of hours when the batteries discharge is 15 hours.
\begin{enumerate}[resume]
    \item Electricity consumption and power ratings
\end{enumerate}
Since different electric appliances require different needs, by researching the average wattage and hours used per device, we can calculate the mean electricity consumption per day. By multiplying the appliance's wattage with the number of hours used, we get the average energy used by the appliance per day. These values will be summed up, which will return the total amount of energy required to power these devices for the day.