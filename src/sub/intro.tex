Batteries are important components in a stand-alone photovoltaic system. It allows the storage of excess solar energy, provides energy for appliances during the night or non-sunny days, and also acts as a stabiliser by providing constant voltage to the loads. Most batteries being used nowadays are lead-acid batteries, though lithium-ion batteries are becoming an increasingly popular option. Batteries are one of the most complex elements in a PV system as many phenomena may occur, such as discharge and efficiency. Many parameters vary during a charge/discharge cycle: voltage, current, resistivity, temperature, et cetera. This leads to difficulties when predicting and simulating such a system. As a result, many homeowners struggled to choose their battery storage system and set up one in their house. Since there are many factors to consider, many stand-alone photovoltaic systems either invest too much money on batteries or too little, which leads to frequent power shortages. Therefore, our team aims to answer the following question: what is the "best" battery storage system for an off-grid home.

To answer this question, we have done in-depth research into the problem, developed multiple mathematical models, and provided a comprehensive and insightful analysis regarding the implementation of battery arrays in solar systems.
