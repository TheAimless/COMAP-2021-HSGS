Battery is an important component in a stand-alone photovoltaic system. It allows the storage of excess solar energy, provides energy for appliances during the night or non-sunny days, and also acts as a stabiliser by providing constant voltage to the loads. Most batteries being used nowadays are lead-acid batteries, though lithium-ion batteries are becoming an increasingly popular option. Batteries, if not, are one of the most complex element in a PV system as many phenomenon occur, such as discharge and efficiency. Many parameters vary during a charge/discharge cycle: voltage, current, resistivity, temperature, etc. This leads to difficulties when predicting and simulating such a system. As a result, many homeowners struggled to choose their battery storage system and set up one of them in their house. Since there are many factors to consider, a lot of stand-alone photovoltaic systems either invest too much money on batteries or too little, which leads to frequent power shortages. Therefore, it is crucial for our team to answer the following question: what is the "best" battery storage system for an off-grid home.

To answer this question, we have done in-depth researches into the problem, developed multiple mathematical models, and provided a comprehensive and insightful analysis regarding the implementation of battery arrays in solar systems.
