\afterpage{
\newgeometry{left=0.5in,right=0.5in,top=0.2in,bottom=0.5in}
\begin{centering}
    Team Control Number

    {\Huge 11682}

    Problem Chosen

    {\Huge \textbf{A}}

    {\Large \textbf{2021}}\\
    {\textbf{HiMCM/MidMCM}}\\
    {\textbf{Summary Sheet}}\\~\\

\end{centering}
\makebox[\textwidth]{\rule{\textwidth}{0.6pt}}
\begin{centering}
    (Your team's summary should be included as the first page of your electronic submission.)\\
    Type a summary of your results on this page. Do not include the name of your school, advisor, or team members on this page.

\end{centering}
Recently, solar power is becoming a popular alternative energy source. However, installing battery banks for use when there is no sunlight remains a big problem for homeowners. The following leads to either insufficient wattage output, which makes generators or an external power grid necessary, or too much money spent on solar batteries. As such, those installing solar panels need to determine the amount and type of battery they should install depending on their needs and budgets. Our mathematical model aims to address the following problem, establish the most cost-effective way to set up battery arrays, in addition to having a look at cement batteries.

To start with, a list of electrical appliances is identified, in addition to any power ratings and number of hours used. These data roughly correlate to the average daily electricity consumption of an American household living in a 1600 square-feet house. Each power ratings are then multiplied by the number of hours used, before summing up to get the total energy used per day.

Then,

Next,

In conclusion, 
\clearpage
\restoregeometry
}
