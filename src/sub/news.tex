\date{\today}
\currentvolume{1}
\currentissue{1}

\SetPaperName{Solar Times:}

\SetHeaderName{Solar Times}

\SetPaperLocation{Florida}
\SetPaperSlogan{``Powered by solar power.''}
\SetPaperPrice{Zero Dollars}
\maketitle
\thispagestyle{fancy}

\begin{multicols}{3}

\headline{Solar batteries: Factors to consider during installation}

As solar power becomes increasingly popular as an alternative energy source, more people are having trouble during the installation of a solar system, especially those who go off-grid. When installing a solar system, there are many factors to consider. However, one of the most important part of your home's electrical system are the batteries. These batteries provide a way to store your hard-earned solar energy, in addition to powering all the necessary electric appliances in your house. Therefore, it is crucial to understand all the details involved during the installation of a battery storage system.

First, there are the energy demands within your household. Every pieces of electrical equipment require electricity in one way or another, so if you plan to go solar you need to calculate your total energy usage per day in kilowatt hours (kWh). By calculating the power consumed and the number of hours used by each electric appliances in your house, you can find out the total energy required to power your home.

After doing the calculations, you need to choose a battery type for your array. Two most popular choices for general home usage are lead-acid batteries and lithium-ion batteries. While lead-acid batteries have existed for a long time and are reliable, there is another option: lithium-ion batteries. These batteries not only offer you more storage per dollar you spend but also let you draw out more electricity from the sun with higher efficiency and depth of discharge (the percentage of power you consume from the battery) without degrading the battery faster.

Next, you can begin estimating the number of batteries required for your house. By either using a solar battery calculator or calculating them on your own but with many variables taken into account, you can find out the number of batteries to purchase. You can also do this with other battery products to find the minimum cost of installation, which can save a lot of money.

In conclusion, you should have set up your own battery array, and you can either hire an electrician to install your home circuit, or you can try to install that on you  
\closearticle

\headline{Cement batteries: A new future?}

Researchers from Gothenburg, Sweden has developed a battery model that incorporates the most common building material: cement. This allows previously unused wall space to be utilised as batteries, potentially revolutionise battery technologies in the future. Through assessing the combinations of many common metals, the team has developed multiple batteries with metals embedded inside a slab of concrete. After several testings, the researchers discovered that a cement battery with the anode made out of nickel and the cathode made out of iron gives the best result: an average energy density of 7 Wh/m$^{2}$, or 0.8 Wh/L. Despite having a lower energy density than traditional solar batteries, researchers Zhang and Tang are optimistic about the possibility of "building rechargeable cement-based batteries on a large scale, with regard to the huge volume of a building." Who knows, if one day you can draw electricity to power your house from 

\closearticle

\end{multicols}
\restoregeometry