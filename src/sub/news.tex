\date{\today}
\currentvolume{1}
\currentissue{1}

\SetPaperName{Solar Times:}

\SetHeaderName{Solar Times}

\SetPaperLocation{Florida}
\SetPaperSlogan{``Powered by solar power.''}
\SetPaperPrice{Zero Dollars}
\maketitle
\thispagestyle{fancy}

\begin{multicols}{3}

\headline{Solar batteries: Factors to consider during installation}

As solar power becomes increasingly popular as an alternative energy source, more people are having trouble during the installation of a solar system, especially those who go off-grid. When installing a solar system, there are many factors to consider. However, one of the most important part of your home's electrical system are the batteries. These batteries provide a way to store your hard-earned solar energy, in addition to powering all the necessary electric appliances in your house. Therefore, it is crucial to understand all the details involved during the installation of a battery storage system.

First, there are the energy demands in your house. By calculating each electric appliances in your house the power consumed and the number of hours used, you can find out the total energy required to power your home. After that, you need to choose a battery type for your array. While there are many options available, two most popular choices are lead-acid batteries and lithium-ion batteries.

Next, you can begin estimating the number of batteries required for your house. By either using a solar battery calculator or calculating them on your own but with many variables taken into account, you can find out the number of batteries to purchase. You can also do this with other battery products to find the minimum cost of installation, which can save a lot of money.

In conclusion, you should have set up your own battery array, and you can either hire an electrician to install your home circuit, or you can try to install that on you  
\closearticle

\headline{Cement batteries: A new future?}

Researchers from Gothenburg, Sweden has developed a battery model that incorporates the most common building material: cement. This allows previously unused wall space to be utilised as batteries, potentially revolutionise battery technologies in the future. Because concrete walls and ceilings take up a large space in the house, they can hold a more electricity than what traditional batteries do. 

\closearticle

\end{multicols}
\restoregeometry