As said above, there are many factors to consider when determining which battery storage system is the "best". Here we have included several important factors that our model use:
\begin{enumerate}
    \item Power rating
\end{enumerate}
%Thực tế, thời gian công suất đạt đỉnh rất ngắn, chỉ tối đa khoảng một phút trong một giờ, quãng thời gian còn lại là thời gian công suất ổn định. Vì vậy, chúng ta có thể tính công suất trung bình khi sử dụng bằng công thức (59 * Pc + Pi), sau đó để kiểm tra độ phù hợp, chúng ta sẽ lấy công suất trung bình chia cho công suất chúng ta sử dụng mỗi giờ, và cân bằng hệ số trung bình (60 phút một giờ, 59 phút dùng công suất ổn định, một phút dùng công suất đỉnh).
When evaluating an electrical device, the time period which it reaches peak power consumption is at most one minute for every hour of usage, with the rest of the time the device consume electricity at a constant rate. Therefore, we can calculate the  
\begin{enumerate}[resume]
    \item Capacity of the battery storage system
\end{enumerate}
If the capacity of the array is too low, the system will either struggle to power the entire home or have the depth-of-discharge higher than the recommended level. In that case, without another back-up power grid, the system will shutdown, leaving electric appliances in the house unusable, and the lifetime of the batteries will decrease.
\begin{enumerate}[resume]
    \item Length of autonomy
\end{enumerate}
In the worst case scenario, there will exist days without sufficient energy to charge the batteries up to full capacity. It is necessary for the system not to shut down during usage. Since weather patterns vary throughout the country, each customer will have a specific demand of length of autonomy.
\begin{enumerate}[resume]
    \item Efficiency of the battery storage system
\end{enumerate}
It is unavoidable that some electricity transmitted through the circuit will be wasted. Therefore, in order to retain as much charge as possible, the efficiency of the system should be as high as can be achieved.