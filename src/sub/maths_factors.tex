After calculating the energy demands for the 1600-square feet house, we can start developing the model. When installing an off-grid battery storage system, there are many variables to be taken into account. The factors considered are listed as follows:

\begin{enumerate}
    \item Cost of installation
\end{enumerate}
There are many homeowners who will spend too much money on the system, therefore wasting both money and energy during the process. In addition, many customers are on a budget, and it is necessary to minimise the installation cost of the battery storage system.
\begin{enumerate}[resume]
    \item Capacity of the battery storage system
\end{enumerate}
If the capacity of the array is too low, the system will either struggle to power the entire home or have the depth-of-discharge higher than the recommended level. In that case, without another back-up power grid, the system will shutdown, leaving electric appliances in the house unusable, and the lifetime of the batteries will decrease.
\begin{enumerate}[resume]
    \item Length of autonomy
\end{enumerate}
In the worst case scenario, there will exist days without sufficient energy to charge the batteries up to full capacity. It is necessary for the system not to shut down during usage. Since weather patterns vary throughout the country, each customer will have a specific demand of length of autonomy.
\begin{enumerate}[resume]
    \item Efficiency of the battery storage system
\end{enumerate}
It is unavoidable that some electricity transmitted through the circuit will be wasted. Therefore, in order to retain as much charge as possible, the efficiency of the system should be as high as can be achieved.